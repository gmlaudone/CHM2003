\documentclass[12pt,a4paper]{report}
\usepackage[utf8]{inputenc}
\usepackage{amsmath}
\usepackage{amsfonts}
\usepackage{amssymb}
\usepackage{graphicx}
\usepackage{chemstyle}
\usepackage{mdframed}
\usepackage{cancel}
\usepackage{tikz}
\usetikzlibrary{matrix}
\addtolength{\topmargin}{-1in}
\setlength{\textwidth}{6.3in}
\setlength{\textheight}{9.5in}
\newcommand*\diff{\mathop{}\!\mathrm{d}}
\newcommand{\overbar}[1]{\mkern 1.5mu\overline{\mkern-1.5mu#1\mkern-1.5mu}\mkern 1.5mu}

% code to hide part of the text leaving the blank space in its place
\ExplSyntaxOn
\box_new:N \l_mypkg_box
\int_new:N \l_mypkg_cleanup_int
\DeclareDocumentCommand{\hideit}{O{1}+m}
  {
    \tex_setbox:D \l_mypkg_box \tex_vbox:D
      {
        #2\par
        \dim_zero:N \tex_baselineskip:D
        \dim_zero:N \tex_lineskip:D
        \dim_zero:N \tex_lineskiplimit:D
        \int_set:Nn \l_mypkg_cleanup_int {#1}
        \mypkg_dismantle_loop:
      }
    \tex_unvbox:D \l_mypkg_box
  }
\cs_new_protected:Npn \mypkg_dismantle_loop:
  {
    \prg_replicate:nn { \l_mypkg_cleanup_int }
      {
        \skip_if_eq:nnT { \tex_lastskip:D } { \c_zero_skip } { \tex_unskip:D }
        \dim_compare:nT { \tex_lastkern:D = \c_zero_dim } { \tex_unkern:D }
        \int_compare:nT { \tex_lastpenalty:D = \c_zero } { \tex_unpenalty:D }
      }
    \skip_if_eq:nnTF { \tex_lastskip:D } { \c_zero_skip }
      {
        \dim_compare:nTF { \tex_lastkern:D = \c_zero_dim }
          {
            \int_compare:nTF { \tex_lastpenalty:D = \c_zero }
              {
                \box_set_to_last:N \l_mypkg_box
                \box_if_empty:NF \l_mypkg_box
                  { \mypkg_dismantle_box: }
              }
              { \mypkg_dismantle_penalty: }
          }
          { \mypkg_dismantle_kern: }
      }
      { \mypkg_dismantle_skip: }
  }
\cs_new_protected:Npn \mypkg_dismantle_skip:
  { \mypkg_dismantle_aux:nN { \tex_vskip:D \skip_use:N \tex_lastskip:D } \tex_unskip:D }
\cs_new_protected:Npn \mypkg_dismantle_kern:
  { \mypkg_dismantle_aux:nN { \tex_kern:D \dim_use:N \tex_lastkern:D } \tex_unkern:D }
\cs_new_protected:Npn \mypkg_dismantle_penalty:
  { \mypkg_dismantle_aux:nN { \tex_penalty:D \int_use:N \tex_lastpenalty:D } \tex_unpenalty:D }
\cs_new_protected:Npn \mypkg_dismantle_box:
  { \mypkg_dismantle_aux:nN { \tex_vbox:D to \dim_eval:n { \box_ht:N \l_mypkg_box + \box_dp:N \l_mypkg_box } { } } \scan_stop: }
\cs_new_protected:Npn \mypkg_dismantle_aux:nN #1#2
  {
    \use:x
      {
        #2
        \mypkg_dismantle_loop:
        #1 \scan_stop:
      }
  }
\ExplSyntaxOff

%conditional compilation for students version of document
\newif\ifstudents
%\studentstrue % comment out to hide text and equations

\begin{document}
   \pagestyle{headings}
   \thispagestyle{plain}
   \newpage
   \noindent
   \begin{center}
   \framebox{
       \vbox{
           \hbox to 6in { {\bf CHM2003 Physical Chemistry 2 \hfill October 2013} }
           \vspace{4mm}
           \hbox to 6in { {\Large \hfill Thermodynamics   \hfill} }
           \vspace{2mm}
           \hbox to 6in { {\it Giuliano Maurizio Laudone \hfill Lecture 4} }
      }
   }
   \end{center}
   \section*{Chemical equilibrium}
   \subsection*{Gibbs-Helmholtz equation}
   For many processes it is useful to know the variation of Gibbs free energy with temperature. During the previous lecture we have seen the fundamental equation: 
   \begin{equation*}
   \diff G= V \diff p-S \diff T
   \end{equation*}
   which gives:
   \begin{equation*} 
   \left(\frac{\partial G}{\partial T}\right)_{p}=-S
   \end{equation*}
   Gibbs free energy $G$ is defined as $H-TS$, which we can rearrange and express as: $S = (H-G)/T$. This equation can be also written as:
   \begin{equation*}
   -S=\frac{G}{T}-\frac{H}{T}
   \end{equation*}
   Since $\left(\frac{\partial G}{\partial T}\right)_{p}=-S$ the previous equation can be written as:
   \begin{equation*}
   \left(\frac{\partial G}{\partial T}\right)_{p} -\frac{G}{T} =-\frac{H}{T}
   \end{equation*}
   We can show that $ \left(\frac{\partial G}{\partial T}\right)_{p} -\frac{G}{T} = T \left(\frac{\partial (G/T)}{\partial T}\right)_{p} $
   \begin{mdframed}
   \textbf{A little bit of maths}\\
   If a function $ f(x) $ is given by the product of 2 functions $g(x)$ and $h(x)$
   \begin{equation*}
   f=g h
   \end{equation*}
   its derivative with respect to $x$ is: 
   \begin{equation*}
   \frac{\diff f}{\diff x} = g \frac{\diff h}{\diff x}+h \frac{\diff g}{\diff x}
   \end{equation*}
   \end{mdframed}
   We can apply the product rule of derivatives to calculate $\left(\frac{\partial (G/T)}{\partial T}\right)_{p}$
   \begin{eqnarray*}
   \left(\frac{\partial (G/T)}{\partial T}\right)_{p} &=& \frac{1}{T} \left(\frac{\partial G}{\partial T}\right)_{p} + G \left(\frac{\partial 1/T}{\partial T}\right)_{p}\\
   &=& \frac{1}{T} \left(\frac{\partial G}{\partial T}\right)_{p} -G \frac{1}{T^{2}}
   \end{eqnarray*}
   By multiplying both sides by $T$:
   \begin{equation*}
   T \left(\frac{\partial (G/T)}{\partial T}\right)_{p} = \left(\frac{\partial G}{\partial T}\right)_{p} - \frac{G}{T}
   \end{equation*}
   which is what we wanted to prove.
   
   We can substitute this result into $\left(\frac{\partial G}{\partial T}\right)_{p} -\frac{G}{T} =-\frac{H}{T}$ and, after dividing both sides by $T$, we obtain:
   \begin{equation*}
   \left(\frac{\partial (G/T)}{\partial T}\right)_{p} = -\frac{H}{T^{2}}
   \end{equation*}
   which is the \textbf{Gibbs-Helmholtz equation}.
   \subsection*{Van't Hoff isotherm}
   The Van't Hoff isotherm correlates the Gibbs free energy of a reaction with its equilibrium constant.
   For a generic chemical reaction at equilibrium:
   \begin{equation*}
   a\mathrm{A}+b\mathrm{B} \rightleftharpoons c\mathrm{C}+d\mathrm{D}
   \end{equation*}
   the equilibrium constant is defined:
   \begin{equation*}
   K_{\mathrm{eq}}=\frac{a_{\mathrm{C}}^{c}\ a_{\mathrm{D}}^{d}}{a_{\mathrm{A}}^{a}\  a_{\mathrm{B}}^{b}}
   \end{equation*}
   where $a_{\mathrm{X}}$ is the activity at equilibrium of the generic X chemical species taking part to the chemical reaction. The activity of a chemical species could be thought of as an \textit{effective concentration}. Using a more generic and compact, but maybe slightly more difficult to follow, mathematical notation we can write the equilibrium constant as:
   \begin{equation*}
   K_{\mathrm{eq}}=\prod\limits_{i} a_{i}^{\nu_{i}}
   \end{equation*}
   where $\nu_{i}$ is the stoichiometric coefficient of species $i$ and it's positive if $i$ is a product and negative is $i$ is a reactant.
   During lecture 2 we have seen that for each of the chemical species we can write:
   \begin{equation*}
   \mu_{i}=\mu_{i}^{\standardstate}+RT \ln a_{i}
   \end{equation*}
   or equivalently:
   \begin{equation*}
   \overbar{G}_{i}=\overbar{G}_{i}^{\standardstate}+RT \ln a_{i}
   \end{equation*}
   where $\mu_{i}$ is the chemical potential of species $i$, $\overbar{G}_{i}$ is the partial molar free energy of specie $i$ (which is by definition equivalent to its chemical potential) and the superscript $\standardstate$ represents the standard conditions.
   
   If we focus our attention to a gas phase reaction we can approximate the activity of each species with its partial pressure, so that the previous equation becomes:
   \begin{equation*}
   \overbar{G}_{i}=\overbar{G}_{i}^{\standardstate}+RT \ln p_{i}
   \end{equation*}
   For a certain species $i$ with stoichiometric coefficient $\nu_{i}$ we can write:
   \begin{equation*}
   \nu_{i}\overbar{G}_{i}-\nu_{i}\overbar{G}_{i}^{\standardstate}=\nu_{i}RT \ln p_{i} = RT \ln p_{i}^{\nu_{i}} 
   \end{equation*}
   We can calculate the difference of free energy between the products and the reactants. In the case of the initial $a\mathrm{A}+b\mathrm{B} \rightleftharpoons c\mathrm{C}+d\mathrm{D}$ this would give:
   \begin{align*}
   (c\overbar{G}_{\mathrm{C}}-c\overbar{G}^{\standardstate}_{\mathrm{C}})+(d\overbar{G}_{\mathrm{D}}-d\overbar{G}^{\standardstate}_{\mathrm{D}})-(a\overbar{G}_{\mathrm{A}}-a\overbar{G}^{\standardstate}_{\mathrm{A}})-(b\overbar{G}_{\mathrm{B}}-b\overbar{G}^{\standardstate}_{\mathrm{B}}) =\\ RT\ln p_{\mathrm{C}}^{c}+RT\ln p_{\mathrm{D}}^{d}-RT\ln p_{\mathrm{A}}^{a}-RT\ln p_{\mathrm{B}}^{b}
   \end{align*}
   The left hand term can be expressed as $\Delta G_{\mathrm{r}}-\Delta G^{\standardstate}_{\mathrm{r}}$ while the left hand side can be expressed as $RT\ln \frac{p_{\mathrm{C}}^{c}\ p_{\mathrm{D}}^{d}}{p_{\mathrm{A}}^{a}\  p_{\mathrm{B}}^{b}}$ or $RT \ln K_{\mathrm{eq}}$.
   Therefore:
   \begin{equation*}
   \Delta G_{\mathrm{r}}-\Delta G^{\standardstate}_{\mathrm{r}}=RT \ln K_{\mathrm{eq}}
   \end{equation*}
   However, we are in a state of equilibrium and $\Delta G_{\mathrm{r}} = 0$ and as a consequence:
   \begin{equation*}
   \Delta G^{\standardstate}_{\mathrm{r}}=-RT \ln K_{\mathrm{eq}}
   \end{equation*}
   \subsection*{Van't Hoff isochore}
   The Van't Hoff isochore allows us to quantify how the equilibrium constant varies with temperature.
   During lecture 3 we have seen that the fundamental equation can be expressed as $\diff G = V\diff p-S\diff T$, which led us to the conclusion that $\left(\frac{\partial G}{\partial T}\right)_{p}=-S$. Therefore for a generic reaction we can write:
   \begin{equation*}
   \left(\frac{\partial \Delta G^{\standardstate}_{\mathrm{r}}}{\partial T}\right)_{p}=-\Delta S^{\standardstate}_{\mathrm{r}}
   \end{equation*}
   We also know that $\Delta G^{\standardstate}_{\mathrm{r}} = \Delta H^{\standardstate}_{\mathrm{r}}-T\Delta S^{\standardstate}_{\mathrm{r}}$, which we can rearrange as $\Delta S^{\standardstate}_{\mathrm{r}} = \frac{\Delta H^{\standardstate}_{\mathrm{r}}-\Delta G^{\standardstate}_{\mathrm{r}}}{T}$.
   Therefore:
   \begin{equation*}
   \left(\frac{\partial \Delta G^{\standardstate}_{\mathrm{r}}}{\partial T}\right)_{p}=-\Delta S^{\standardstate}_{\mathrm{r}} = \frac{\Delta G^{\standardstate}_{\mathrm{r}}-\Delta H^{\standardstate}_{\mathrm{r}}}{T}
   \end{equation*}
   But from the Van't Hoff isotherm we know that $\Delta G^{\standardstate}_{\mathrm{r}} =-RT \ln K_{\mathrm{eq}}$. Differentiating this by parts:
   \begin{equation*}
   \frac{\partial\Delta G^{\standardstate}_{\mathrm{r}}}{\partial T}=-R\ln K_{\mathrm{eq}}-RT \frac{\partial \ln K_{\mathrm{eq}}}{\partial T}
   \end{equation*}
   Therefore:
   \begin{equation*}
  \cancel{-R\ln K_{\mathrm{eq}}}-RT \frac{\partial \ln K_{\mathrm{eq}}}{\partial T}=\frac{\cancel{\Delta G^{\standardstate}_{\mathrm{r}}}-\Delta H^{\standardstate}_{\mathrm{r}}}{T}
   \end{equation*}
   from which:
   \begin{equation*}
   \frac{\Delta H^{\standardstate}_{\mathrm{r}}}{T}=RT \frac{\partial \ln K_{\mathrm{eq}}}{\partial T}
   \end{equation*}   
   and finally:
   \begin{equation*}
   \frac{\partial \ln K_{\mathrm{eq}}}{\partial T} = \frac{\Delta H^{\standardstate}_{\mathrm{r}}}{RT^{2}}
   \end{equation*} 
   which is the Van't Hoff equation.
  
   We can integrate this differential equation and, assuming that the enthalpy change during the reaction does not depend on the value of temperature, we obtain:
    \begin{equation*}
   \ln K_{\mathrm{eq,2}}=\ln K_{\mathrm{eq,1}}-\frac{\Delta H^{\standardstate}_{\mathrm{r}}}{R}\left(\frac{1}{T_{2}}-\frac{1}{T_{1}}\right)
   \end{equation*} 
\end{document}